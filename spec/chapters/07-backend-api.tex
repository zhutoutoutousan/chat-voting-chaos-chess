\chapter{Backend API Design}

\section{Overview}

The Nest.js backend provides a comprehensive RESTful API and WebSocket gateway for all platform functionality. The API follows REST principles with clear resource-based endpoints and consistent response formats.

\section{API Architecture}

\subsection{Module Organization}

The backend is organized into feature modules:

\begin{itemize}
    \item \textbf{Auth Module}: Authentication and authorization
    \item \textbf{Users Module}: User management
    \item \textbf{Games Module}: Game operations
    \item \textbf{Chess Module}: Chess engine and move validation
    \item \textbf{Ratings Module}: ELO rating calculations
    \item \textbf{Academy Module}: Learning content
    \item \textbf{Chat Module}: Chat functionality
    \item \textbf{Voting Module}: Chaos mode voting
    \item \textbf{Payments Module}: Stripe integration
    \item \textbf{Notifications Module}: User notifications
\end{itemize}

\subsection{API Versioning}

\begin{itemize}
    \item \textbf{Base URL}: \texttt{/api/v1}
    \item \textbf{Version Header}: \texttt{API-Version: v1}
    \item \textbf{Backward Compatibility}: Maintained for at least one major version
\end{itemize}

\section{Authentication Endpoints}

\subsection{POST /api/v1/auth/register}

Register a new user account.

\textbf{Request Body:}
\begin{lstlisting}[language=json]
{
  "email": "user@example.com",
  "username": "chessplayer",
  "password": "securepassword",
  "name": "Chess Player"
}
\end{lstlisting}

\textbf{Response:}
\begin{lstlisting}[language=json]
{
  "user": {
    "id": "clx...",
    "email": "user@example.com",
    "username": "chessplayer"
  },
  "accessToken": "eyJhbGc...",
  "refreshToken": "eyJhbGc..."
}
\end{lstlisting}

\subsection{POST /api/v1/auth/login}

Authenticate user and return tokens.

\textbf{Request Body:}
\begin{lstlisting}[language=json]
{
  "email": "user@example.com",
  "password": "securepassword"
}
\end{lstlisting}

\subsection{POST /api/v1/auth/oauth/:provider}

Initiate OAuth flow with specified provider (google, github, discord).

\subsection{POST /api/v1/auth/refresh}

Refresh access token using refresh token.

\subsection{POST /api/v1/auth/logout}

Invalidate current session.

\section{User Endpoints}

\subsection{GET /api/v1/users/me}

Get current user profile.

\textbf{Response:}
\begin{lstlisting}[language=json]
{
  "id": "clx...",
  "email": "user@example.com",
  "username": "chessplayer",
  "name": "Chess Player",
  "avatar": "https://...",
  "bio": "...",
  "ratings": {
    "overall": 1500,
    "blitz": 1450,
    "rapid": 1520
  },
  "isPremium": false,
  "createdAt": "2024-01-01T00:00:00Z"
}
\end{lstlisting}

\subsection{PUT /api/v1/users/me}

Update current user profile.

\subsection{GET /api/v1/users/:id}

Get public user profile by ID.

\subsection{GET /api/v1/users/:id/stats}

Get user statistics.

\textbf{Response:}
\begin{lstlisting}[language=json]
{
  "gamesPlayed": 150,
  "wins": 75,
  "losses": 50,
  "draws": 25,
  "winRate": 50.0,
  "averageRating": 1500,
  "peakRating": 1650
}
\end{lstlisting}

\section{Game Endpoints}

\subsection{POST /api/v1/games}

Create a new game.

\textbf{Request Body:}
\begin{lstlisting}[language=json]
{
  "opponentId": "clx...",
  "timeControl": "300+0",
  "isRated": true,
  "isChaosMode": false,
  "isPublic": true
}
\end{lstlisting}

\textbf{Response:}
\begin{lstlisting}[language=json]
{
  "id": "game-id",
  "whitePlayer": {...},
  "blackPlayer": {...},
  "status": "WAITING",
  "timeControl": "300+0",
  "createdAt": "2024-01-01T00:00:00Z"
}
\end{lstlisting}

\subsection{GET /api/v1/games}

List games with filtering and pagination.

\textbf{Query Parameters:}
\begin{itemize}
    \item \texttt{status}: Filter by game status
    \item \texttt{userId}: Filter by player
    \item \texttt{isRated}: Filter rated games
    \item \texttt{page}: Page number
    \item \texttt{limit}: Items per page
\end{itemize}

\subsection{GET /api/v1/games/:id}

Get game details.

\textbf{Response:}
\begin{lstlisting}[language=json]
{
  "id": "game-id",
  "whitePlayer": {...},
  "blackPlayer": {...},
  "status": "IN_PROGRESS",
  "currentFen": "rnbqkbnr/pppppppp/8/8/8/8/PPPPPPPP/RNBQKBNR w KQkq - 0 1",
  "moves": [
    {
      "moveNumber": 1,
      "move": "e2e4",
      "san": "e4",
      "isWhiteMove": true,
      "timestamp": "2024-01-01T00:00:00Z"
    }
  ],
  "whiteTimeLeft": 285000,
  "blackTimeLeft": 300000,
  "startedAt": "2024-01-01T00:00:00Z"
}
\end{lstlisting}

\subsection{POST /api/v1/games/:id/moves}

Make a move in a game.

\textbf{Request Body:}
\begin{lstlisting}[language=json]
{
  "move": "e2e4",
  "from": "e2",
  "to": "e4"
}
\end{lstlisting}

\textbf{Response:}
\begin{lstlisting}[language=json]
{
  "success": true,
  "move": {
    "moveNumber": 1,
    "move": "e2e4",
    "fen": "rnbqkbnr/pppppppp/8/8/4P3/8/PPPP1PPP/RNBQKBNR b KQkq e3 0 1",
    "isWhiteMove": true
  },
  "gameStatus": "IN_PROGRESS",
  "result": null
}
\end{lstlisting}

\subsection{POST /api/v1/games/:id/resign}

Resign from a game.

\subsection{POST /api/v1/games/:id/draw-offer}

Offer a draw.

\subsection{POST /api/v1/games/:id/draw-accept}

Accept a draw offer.

\subsection{GET /api/v1/games/:id/analysis}

Get game analysis with engine evaluation.

\section{Rating Endpoints}

\subsection{GET /api/v1/ratings/me}

Get current user's ratings.

\textbf{Response:}
\begin{lstlisting}[language=json]
{
  "overall": {
    "rating": 1500,
    "gamesPlayed": 100,
    "wins": 50,
    "losses": 30,
    "draws": 20
  },
  "blitz": {...},
  "rapid": {...},
  "classical": {...}
}
\end{lstlisting}

\subsection{GET /api/v1/ratings/leaderboard}

Get leaderboard for a rating type.

\textbf{Query Parameters:}
\begin{itemize}
    \item \texttt{type}: Rating type (BLITZ, RAPID, etc.)
    \item \texttt{page}: Page number
    \item \texttt{limit}: Items per page
\end{itemize}

\subsection{GET /api/v1/ratings/:userId/history}

Get rating history for a user.

\textbf{Response:}
\begin{lstlisting}[language=json]
{
  "ratingType": "BLITZ",
  "history": [
    {
      "rating": 1500,
      "change": 0,
      "gameId": "game-id",
      "timestamp": "2024-01-01T00:00:00Z"
    }
  ]
}
\end{lstlisting}

\section{Academy Endpoints}

\subsection{GET /api/v1/academy/courses}

List all academy courses.

\subsection{GET /api/v1/academy/courses/:id}

Get course details.

\subsection{GET /api/v1/academy/courses/:id/lessons}

Get lessons for a course.

\subsection{GET /api/v1/academy/lessons/:id}

Get lesson content.

\subsection{POST /api/v1/academy/lessons/:id/complete}

Mark lesson as completed.

\subsection{GET /api/v1/academy/progress}

Get user's academy progress.

\subsection{GET /api/v1/academy/puzzles/daily}

Get daily puzzle.

\subsection{POST /api/v1/academy/puzzles/:id/solve}

Submit puzzle solution.

\section{Chat Endpoints}

\subsection{GET /api/v1/chat/games/:gameId/messages}

Get chat messages for a game.

\textbf{Query Parameters:}
\begin{itemize}
    \item \texttt{limit}: Number of messages
    \item \texttt{before}: Get messages before timestamp
\end{itemize}

\subsection{POST /api/v1/chat/games/:gameId/messages}

Send a chat message (also via WebSocket).

\textbf{Request Body:}
\begin{lstlisting}[language=json]
{
  "message": "Great move!",
  "messageType": "TEXT"
}
\end{lstlisting}

\section{Voting Endpoints (Chaos Mode)}

\subsection{GET /api/v1/voting/games/:gameId/current-round}

Get current voting round information.

\textbf{Response:}
\begin{lstlisting}[language=json]
{
  "round": 1,
  "suggestedMoves": [
    {"move": "e2e4", "votes": 50},
    {"move": "d2d4", "votes": 30}
  ],
  "timeRemaining": 15000,
  "isActive": true
}
\end{lstlisting}

\subsection{POST /api/v1/voting/games/:gameId/vote}

Submit a vote for a move.

\textbf{Request Body:}
\begin{lstlisting}[language=json]
{
  "move": "e2e4"
}
\end{lstlisting}

\section{Payment Endpoints}

\subsection{POST /api/v1/payments/create-checkout}

Create Stripe checkout session.

\textbf{Request Body:}
\begin{lstlisting}[language=json]
{
  "tier": "PREMIUM",
  "successUrl": "https://...",
  "cancelUrl": "https://..."
}
\end{lstlisting}

\textbf{Response:}
\begin{lstlisting}[language=json]
{
  "sessionId": "cs_...",
  "url": "https://checkout.stripe.com/..."
}
\end{lstlisting}

\subsection{GET /api/v1/payments/subscription}

Get current user's subscription.

\subsection{POST /api/v1/payments/cancel-subscription}

Cancel current subscription.

\subsection{POST /api/v1/payments/webhook}

Stripe webhook endpoint (handles subscription events).

\section{Error Handling}

\subsection{Error Response Format}

All errors follow a consistent format:

\begin{lstlisting}[language=json]
{
  "statusCode": 400,
  "message": "Validation failed",
  "error": "Bad Request",
  "details": [
    {
      "field": "email",
      "message": "Email is required"
    }
  ],
  "timestamp": "2024-01-01T00:00:00Z",
  "path": "/api/v1/auth/register"
}
\end{lstlisting}

\subsection{HTTP Status Codes}

\begin{itemize}
    \item \textbf{200 OK}: Successful request
    \item \textbf{201 Created}: Resource created
    \item \textbf{400 Bad Request}: Invalid request
    \item \textbf{401 Unauthorized}: Authentication required
    \item \textbf{403 Forbidden}: Insufficient permissions
    \item \textbf{404 Not Found}: Resource not found
    \textbf{409 Conflict}: Resource conflict
    \item \textbf{422 Unprocessable Entity}: Validation error
    \item \textbf{500 Internal Server Error}: Server error
\end{itemize}

\section{Request Validation}

All endpoints use DTOs (Data Transfer Objects) with class-validator for validation:

\begin{lstlisting}[language=typescript]
export class CreateGameDto {
  @IsString()
  @IsNotEmpty()
  opponentId: string;

  @IsString()
  @IsNotEmpty()
  timeControl: string;

  @IsBoolean()
  @IsOptional()
  isRated?: boolean;
}
\end{lstlisting}

\section{Rate Limiting}

\begin{itemize}
    \item \textbf{General API}: 100 requests per minute per IP
    \item \textbf{Authentication}: 5 requests per minute per IP
    \item \textbf{Game Moves}: 10 moves per second per user
    \item \textbf{Chat}: 20 messages per minute per user
\end{itemize}

\section{Pagination}

All list endpoints support pagination:

\textbf{Query Parameters:}
\begin{itemize}
    \item \texttt{page}: Page number (default: 1)
    \item \texttt{limit}: Items per page (default: 20, max: 100)
\end{itemize}

\textbf{Response Format:}
\begin{lstlisting}[language=json]
{
  "data": [...],
  "meta": {
    "page": 1,
    "limit": 20,
    "total": 100,
    "totalPages": 5,
    "hasNext": true,
    "hasPrev": false
  }
}
\end{lstlisting}
